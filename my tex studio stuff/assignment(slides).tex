\documentclass{article}

\title{HISTORIC VIEWS OF PROGRAMMING LANGUAGES}
\date{2021-10-2}
\author{Nwoye Kamsiyochukwu}
\begin{document}
	\pagenumbering{gobble}
	\maketitle
	\newpage
	\pagenumbering{arabic}
	
	\section{HISTORY OF PYTHON;}
	Python is an interpreted, general-purpose and high-level programming language developed by Guido van Rossum in the late 1980s at the National Research Institute for Mathematics and Computer Science in Amsterdam.
	It was derived or rather considered a successor to ABC programming language which was designed to teach programming language in the early 1980s by Lambert Meertens, Leo Geurts and others at the CWI(Centrum Wiskunde and Informatica, a National Research Institute).
	When compared to other programming language it is well known for its simple programming syntax, code readability and English-like commands, due to this, it is the most preferred and most popular programming language. 
	It is used in  development, scripting and software testing, apps such as Instagram, search engines such as google and games like battlefield made use of python. 
	The programming language offers may programming libraries like NumPy, Pandas, Matplotlib etc. There are IDEs in association with python sch as IDLE, PyCharm, Visual studio code, Atom etc. Also PERL happens to be similar programming language to python.
	

	\section{HISTORY OF JAVA;}
	It was developed by a team lead by James Gosling at Sun Microsystems. It was originally called “oak” and was designed in 1991 but later renamed in 1995 to what we now know as JAVA.
	It was originally designed for use in embedded chips in consumer electronic appliances but in 1995 after being named JAVA it was redesigned for developing internet application. It is also used in developing codes to communicate with and control the robotic rover(vehicles) on mars.
	It is simple, object oriented, portable and dynamic.
	IDEs associated with JAVA are Eclipse, Kite, IntelliJ IDEA etc.
	JAVA as a programming language can be used to develop web application (amazon), mobile application, scientific application, web servers and application servers.
	\newpage
	\section{HISTORY OF HTML;}
	It is a programming language that was developed by Tim Berners Lee (also the person that developed the World Wide Web) in 1993.It is a markp language designed to link web pages.
	It was developed due to the fact that he was frustrated from having to log onto different compters to find different information from them and sought to seek a solution to it.
	He figured that there must be a way to hop from one set of information to another thats on different computers. This concept of a hyper-text system (connected with the networking technology and protocols needed to pass information between computers) would go on to form the basis for the fundamental language of the world wide web (HTML).
	It is used in web pages development, web document creation, internet navigation etc.
	IDEs associated with HTML are RJ TextEd, NetBeans etc.
	\newpage
	\section{FORTRAN;}
	It is a general-purpose, compiled programming language developed by IBM(International Business Machines Cooperation) in 1950 for scientific and engineering applications.
	Its name was derived from FORmula TRANslating system .
	Its is basically used for scientific and engineering computing.
	IDEs associated with FORTRAN is GNUplot, geany etc.
	
	
	
\end{document}